% LaTeX Arquivo para Currículos
% Adicione na mesma pasta o arquivo: res.cls

\documentclass{res}

\usepackage[english]{babel}
\usepackage[utf8x]{inputenc}
\usepackage{fancyhdr}  % manter 2 linhas no cabeçalho
\usepackage{graphicx}  % adicionar imagens
\usepackage{url}       % adicionar páginas

\renewcommand{\headrulewidth}{0pt} % suprime a linha do cabeçalho
\setlength{\headsep}{24pt}  % espaço entre o cabeçalho e o texto
\setlength{\headheight}{24pt} % permite a 2a. linha do cabeçalho
\pagestyle{fancy} % estilo da pagina
\rhead{ {\it Z. Zeza}\\{\it p. \thepage} } % cabeçalho da 2a. página
\cfoot{} % rodape vazio
\topmargin=-0.5in % start text higher on the page

\begin{document}

\thispagestyle{empty} % Esta página não possui cabeçalho
\name{ZEZINHO DA ZEZA\\[10pt]}
\address{{\bf Cidade} -- Bairro \\
  Tel. (61) 9999.9999 \\  Tel.Contato (61) 9999.9999}      
                                      
\begin{resume}
 
\section{RESUMO PROFISSIONAL}
\vspace{8pt}
Detalhe aqui o tipo de profissional que você é, o que já conseguiu na carreira 
e o que ainda pode conseguir. 
 
\section{HABILIDADES}
\vspace{18pt} 
\begin{description}
 \item[Linguagens] Java, Python, Cobol, \ldots
 \item[Para Web] JavaScript, JQuery, Meteor.js, \ldots 
 \item[Banco de Dados] MongoDB, Oracle, \ldots 
 \item[Específicos] Padrões de Projeto, Orientação a Objetos, IoT, \ldots 
\end{description}

\section{CERTIFICAÇÕES E TÍTULOS}
\vspace{8pt}
{\sl Certificação 1}, Instituição \hfill Mês/Ano \\
{\sl Certificação 2}, Instituição \hfill Mês/Ano \\
{\sl Certificação 3}, Instituição \hfill Mês/Ano \\
{\sl Certificação 4}, Instituição \hfill Mês/Ano

\section{EDUCAÇÃO}
\vspace{8pt} 
{\sl Graduacao}: Nome da Instituição -- Título     \hfill Mês/Ano \\
{\sl Pós-Graduacao}: Nome da Instituição -- Título \hfill Mês/Ano
  
\section{IDIOMAS} 
\vspace{18pt}
\begin{description}
 \item[Inglês] -- Nível avançado tanto para leitura como para escrita.
 \item[Espanhol] -- Nível avançado tanto para leitura como para escrita.
\end{description}
 
\section{EXPERIÊNCIA EM TRABALHOS VOLUNTÁRIOS E CAUSAS}
\vspace{8pt} 
{\sl Empresa} -- Cargo \hfill Mes/Ano - Mes/Ano
\begin{itemize}
  \item Descrição das atividades.
\end{itemize}

\section{HISTÓRICO PROFISSIONAL} % Da mais nova para mais antiga
\vspace{8pt}

% Mais Recente
{\sl Empresa 01} \hfill   Mes/Ano - Atual
\begin{itemize}
  \item Descrição das atividades. 
\end{itemize}

% Anterior 1
{\sl Empresa 02} \hfill   Mes/Ano - Mes/Ano
\begin{itemize}
  \item Descrição das atividades. 
\end{itemize}
 
% Anterior 2
{\sl Empresa 03} \hfill   Mes/Ano - Mes/Ano
\begin{itemize}
  \item Descrição das atividades. 
\end{itemize}

% Anterior 3
{\sl Empresa 04} \hfill   Mes/Ano - Mes/Ano
\begin{itemize}
  \item Descrição das atividades. 
\end{itemize}

\section{MAIORES INFORMAÇÕES}
\vspace{18pt} 
\begin{description}
 \item[Página Pessoal] \url{http://xpto.com.br/zezinho}
 \item[Blog Pessoal] \url{http://xpto.com.br/zezinho}
 \item[Perfil no Linkedin] \url{http://xpto.com.br/zezinho}
\end{description}

\end{resume} 
\end{document}