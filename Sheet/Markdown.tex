\documentclass{article}
%%%%%%%%%%%%%%%%%%%%%%%%%%%%%%%%%%%%%%%%%%%%%%%%%%%%%%%
% Markdown Cheat Sheet
%%%%%%%%%%%%%%%%%%%%%%%%%%%%%%%%%%%%%%%%%%%%%%%%%%%%%%%

% Identificação
\newcommand{\pbtitulo}{\huge{\textbf{Markdown ou Markup Language}}}
\newcommand{\pbversao}{1.0}

\usepackage{sty/cheatsheet}

\begin{document}

\begin{center}{\pbtitulo}\\
{\large Fernando Anselmo - Versão \pbversao}
\end{center}

\begin{multicols*}{3}

\tikzstyle{mybox} = [draw=contorno, fill=white, very thick,
    rectangle, rounded corners, inner sep=10pt, inner ysep=10pt]
\tikzstyle{fancytitle} =[fill=DarkBlue, text=white, font=\bfseries]

%------------ Básicos ---------------
\begin{tikzpicture}
  \node [mybox] (box){%
    \begin{minipage}{0.3\textwidth} \vspace{0.5em}
      \textit{Markdown} é uma linguagem simples de marcação criada por \textbf{John Gruber} e \textbf{Aaron Swartz}.  \\[2mm]
      Parágrafos: Para criar um novo parágrafo, basta adicionar uma linha vazia. \\[2mm]
	  Cabeçalhos (sequencia de dashs): \\
      \codigo{\# cab1 \#\# cab2 ... \#\#\#\#\# cab5} \\[2mm]
	  Negrito: \\
	  \codigo{**Texto em Negrito** \\
 	    \_\_Outro em Negrito\_\_} \\[2mm]
	  Itálico: \\
	  \codigo{*Texto em Itálico* \\
	  \_Outro em Itálico\_} \\[2mm]
  	  Negrito e Itálico ( ou combinação de * e \_ ): \\
      \codigo{***Texto em Negrito e Itálico*** \\
      \_\_\_Outro em Negrito e Itálico\_\_\_} \\[2mm]
  	  Cortado: \\
	  \codigo{$\string~\string~$Texto em Cortado$\string~\string~$} \\[2mm]
	  Citação: \\
      \codigo{> citação} \\[2mm]
	  Aninhar Citação: \\
      \codigo{> citação \\
       > \\
       >> citação interna} \\[2mm]
	  Linha Horizontal ( - ou * ou \_ ): \\
      \codigo{---}
    \end{minipage}
  };
  \node[fancytitle, right=10pt] at (box.north west) {Básicos};
\end{tikzpicture}

%------------ Âncoras ---------------
\begin{tikzpicture}
  \node [mybox] (box){%
	\begin{minipage}{0.3\textwidth} \vspace{0.5em}
	  Âncora para site: \\
	  \codigo{$\string[$Texto$\string]$(url "Dica")} \\[2mm]
	  Âncora de rodapé: \\
      \codigo{$\string[$Texto$\string]\string[$1$\string]$ \\
        ... \\
        $\string[$1$\string]$: url} \\[2mm]
      Âncoras de Imagem na mesma linha: \\
	  \codigo{Texto !$\string[$Texto$\string]$(/nome.png "Dica")}
	\end{minipage}
  };
  \node[fancytitle, right=10pt] at (box.north west) {Âncoras};
\end{tikzpicture}

%------------ Listas ---------------
\begin{tikzpicture}
  \node [mybox] (box){%
    \begin{minipage}{0.3\textwidth} \vspace{0.5em}
      Lista Não Ordenada (* ou -): \\
      \codigo{* Item 1 \\
      	* Item 2 \\
        * Item 3} \\[2mm]
      Lista Ordenada: \\
      \codigo{1 Item 1 \\
   	    2 Item 2 \\
        3 Item 3} \\[2mm]
      Para sublistas basta adicionar espaços \\[2mm]
      Lista Tarefas: \\
      \codigo{-[x] Item Resolvido \\
	    -[ ] Item a Resolver \\
	    -[ ] Item a Resolver} \\[2mm]
      Lista Definida: \\
      \codigo{Primeiro Termo \\
      : Esta é a definição do primeiro termo. \\
      Segundo Termo \\
      : Esta é a definição do segundo termo. \\
      : Como é muito grande continua aqui.}
    \end{minipage}
  };
  \node[fancytitle, right=10pt] at (box.north west) {Listas};
\end{tikzpicture}

%---------- Estruturação do Texto ------------------
\begin{tikzpicture}
  \node [mybox] (box){%
    \begin{minipage}{0.3\textwidth} \vspace{0.5em}
	  Nota de Rodapé: \\
      \codigo{Parágrafo com uma marcação. $\string[\string^$1$\string]$ \\
      	... \\
      	$\string[\string^$1$\string]$: Este é o rodapé.} \\[2mm]
      Tabela: \\
      \codigo{| Coluna 1 |  Coluna 2 | \\
   	    |----------|-----------| \\
        |  Item A  | Descrição | \\
 	    |  Item B  | Descrição |} \\[2mm]
	  Bloco de Código ( $\grave{ }$ ou $\string~$ ): \\
      \codigo{ $\grave{ }$ $\grave{ }$ $\grave{ }$ \\
   	    ; saida \\
	    mov eax, 1 \\
	    mov ebx, 0 \\
	    int 0x80 \\
	    $\grave{ }$ $\grave{ }$ $\grave{ }$ }
    \end{minipage}
  };
  \node[fancytitle, right=10pt] at (box.north west) {Estruturação do Texto};
\end{tikzpicture}

%------------ Perfumaria ---------------------
\begin{tikzpicture}
  \node [mybox] (box){%
	\begin{minipage}{0.3\textwidth} \vspace{0.5em}
  	  Emojis1: Copiar um emoji de uma fonte como a Emojipedia (\url{https://emojipedia.org/}) e colá-lo. \\[2mm]
  	  Emojis2: \textbf{Rafael Xavier} disponibiliza uma coleção em \url{https://gist.github.com/rxaviers/7360908}. Por exemplo: \\
	    \codigo{Esse código :heart\_eyes:} parabéns. \\[2mm]
	  Modo Livre das URL: Ao colocar qualquer URL no código automaticamente é criado um hiperlink, caso não deseje isso: \\
      \codigo{$\grave{ }$ url $\grave{ }$} \\[2mm]
      Abreviação: \\
        \codigo{Markdown é mais simples que HTML. \\
        ... \\
        *[HTML]: HyperText Markup Language} \\[2mm]
      Colocar trecho de código na mesma linha: \\
      \codigo{Texto $\grave{ }$ codigo $\grave{ }$ mais texto} \\[2mm]
      Endereço de eMail: \\
      \codigo{<fake@example.com>} \\[2mm]
      Alinhamento da coluna de tabela é realizado na linha que separa o cabeçalho basta colocar 2 pontos na posição desejada, por exemplo: \\
      \codigo{| :---  | :----: | ---: |} \\[2mm]
      Colocar 3 tipos de código (em abas): \\
      \codigo{ $\string~$$\string~$$\string~$css \\
      Código em CSS. \\
      $\string~$$\string~$$\string~$ \\
      $\string~$$\string~$$\string~$javascript \\
      Código em JavaScript. \\
      $\string~$$\string~$$\string~$ \\
      $\string~$$\string~$$\string~$html \\
      Código em HTML. \\
      $\string~$$\string~$$\string~$} \\[2mm]
      Para exibir um caractere protegido pela linguagem que, de outra forma, seria usado para formatar texto no documento Markdown, adicione uma barra invertida ( $\setminus$ ) na frente do caractere.
	\end{minipage}
  };
  \node[fancytitle, right=10pt] at (box.north west) {Perfumaria};
\end{tikzpicture}

Guia do Markdown: \url{https://www.markdownguide.org/}

\end{multicols*}
\end{document}