\documentclass{article}
%%%%%%%%%%%%%%%%%%%%%%%%%%%%%%%%%%%%%%%%%%%%%%%%%%%%%%%
% Django Cheat Sheet
%%%%%%%%%%%%%%%%%%%%%%%%%%%%%%%%%%%%%%%%%%%%%%%%%%%%%%%

% Identificação
\newcommand{\pbtitulo}{\huge{\textbf{Django}}}
\newcommand{\pbversao}{1.0}

\usepackage{sty/cheatsheet}

\begin{document}

\begin{center}{\pbtitulo}\\
{\large Fernando Anselmo - Versão \pbversao}
\end{center}

\begin{multicols*}{3}

\tikzstyle{mybox} = [draw=contorno, fill=white, very thick,
    rectangle, rounded corners, inner sep=10pt, inner ysep=10pt]
\tikzstyle{fancytitle} =[fill=DarkBlue, text=white, font=\bfseries]

%------------ Inicializações ---------------
\begin{tikzpicture}
  \node [mybox] (box){%
    \begin{minipage}{0.3\textwidth} \vspace{0.5em}
	  Framework para aplicações web gratuito e de código aberto, escrito em Python, que auxilia a desenvolver sites de forma mais rápida e fácil. \\[2mm]
	  Instalar o Django (requer Python 3.x instalado): \\
      \codigo{pip install django} \\[2mm]
	  Criar um projeto: \\
	  \codigo{django-admin startproject [nome]} \\[2mm]
	  Iniciar o servidor (na pasta do projeto): \\
	  \codigo{python manage.py runserver [IP:Porta]} \\[2mm]
	  Criar um administrador do Site: \\
	  \codigo{python manage.py createsuperuser} \\[2mm]
	  Endereço de Administração: \\
	  \url{http://127.0.0.1:8000/admin}
    \end{minipage}
  };
  \node[fancytitle, right=10pt] at (box.north west) {Passos Iniciais};
\end{tikzpicture}

%------------ Models ---------------
\begin{tikzpicture}
  \node [mybox] (box){%
    \begin{minipage}{0.3\textwidth} \vspace{0.5em}
      Tipos de campo mais comuns: \\
      \codigo{IntegerField, DecimalField, CharField, TextField, EmailField, URLField, BooleanField, DateField, TimeField, DateTimeField, FileField, ImageField} \\[2mm]
      Exemplo para um campo tipo Char: \\
	  \codigo{models.CharField(max\_length=10, null=True, blank=True, default=True)} \\[2mm]
	  Outros parâmetros: \\
	  \codigo{required=boolean, label=string, help\_text=string, initial=valorOuFuncao, validators=listaDeValores, swidget=Widget, error\_messages=dict, localize=boolean} \\[2mm]
      Relacionamentos: \\
	  \codigo{oreignKey, OneToOne e ManyToMany} \\[2mm]
      Alterações no modelo: \\
      \codigo{python manage.py makemigrations \\
      python manage.py migrate} \\[2mm]
      Verificar necessidade de migração: \\
      \codigo{python manage.py showmigrations}
    \end{minipage}
  };
  \node[fancytitle, right=10pt] at (box.north west) {Modelos};
\end{tikzpicture}

%------------ Site ---------------------
\begin{tikzpicture}
  \node [mybox] (box){%
    \begin{minipage}{0.3\textwidth} \vspace{0.5em}
	  Criar uma Aplicação: \\
      \codigo{python manage.py startapp [nome]} \\[2mm]
      Adicionar a aplicação para o Administrador: \\
      1. No arquivo settings.py \\
      \codigo{INSTALLED\_APPS = [ ..., \aspas{[nome].apps.[nome]Config}, ]} \\
      2. No arquivo urls.py \\
  	  \codigo{urlpatterns = [ ..., path(\aspas{}, include(\aspas{[nome].urls})), ]}
    \end{minipage}
  };
  \node[fancytitle, right=10pt] at (box.north west) {Aplicação};
\end{tikzpicture}

%------------ Templates ---------------------
\begin{tikzpicture}
  \node [mybox] (box){%
    \begin{minipage}{0.3\textwidth} \vspace{0.5em}
	  Comandos para controlar os códigos nas Templates: \\
  	  Bloco: \codigo{\{\% block nome \%\}\{\% endblock \%\}} \\
  	  CSRF: \codigo{\{\% csrf\_token \%\}} \\
  	  Depurar: \codigo{\{\% debug \%\}} \\
  	  Extends: \codigo{\{\% extends \aspas{base.html} \%\}} \\
  	  Filtro: \codigo{\{\% filter \%\}\{\% endfilter \%\}} \\
  	  Primeiro: \codigo{\{\% firstof \%\}} \\
  	  Repetição: \codigo{\{\% for in \%\}\{\% empty \%\}\{\% endfor \%\}} \\
  	  Decisão: \\
  	  \codigo{\{\% if \%\}\{\% elif \%\}\{\% else \%\}\{\% endif \%\}} \\
  	  \codigo{\{\% ifchanged \%\}\{\% endifc­hanged\%\}} \\
  	  \codigo{\{\% ifequal \%\}\{\% endifequal \%\}} \\
  	  \codigo{\{\% ifnotequal \%\}\{\% endifn­otequal\%\}} \\
  	  Inclusão: \codigo{\{\% include \%\}} \\
  	  Ler: \codigo{\{\% load \%\}} \\
  	  Data Atual: \codigo{\{\% now \aspas{jS F Y H:i} \%\}} \\
  	  Regrupar: \codigo{\{\% regroup by as \%\}} \\
  	  spaceless: \codigo{\{\% spaceless \%\}\{\% endspa­celess\%\}} \\
  	  SSI: \codigo{\{\% ssi \%\}} \\
  	  Contexto Estático: \codigo{\{\% static \%\}} \\
  	  Tag: \codigo{\{\% templa­tetag\%\}} \\
  	  Rrans: \codigo{\{\% trans \%\}}
	  URL: \codigo{\{\% url \%\}} \\
  	  Verbatim: \codigo{\{\% verbatim \%\}\{\% endver­batim\%\}} \\
  	  Ratio: \codigo{\{\% widthratio \%\}} \\
  	  Objeto: \codigo{\{\% with as \%\}\{\% endwith \%\}}
	\end{minipage}
  };
  \node[fancytitle, right=10pt] at (box.north west) {Templates};
\end{tikzpicture}

%------------ Request/Response ---------------------
\begin{tikzpicture}
  \node [mybox] (box){%
	\begin{minipage}{0.3\textwidth} \vspace{0.5em}
	  Conteúdo do \textbf{HttpRequest}: \\
	  \codigo{\_\_iter\_\_(), body, build\_absolute\_uri(path), COOKIES. encoding. GET, POST, REQUEST, FILES, get\_full\_path(), get\_host(), get\_signed\_cookie(key), is\_ajax(), is\_secure(), META, method, path, path\_info, read(size=None), readline(), readlines(), session, urlconf, user} \\[2mm]
	  Conteúdo do \textbf{HttpResponse}: \\
      \codigo{\_\_init\_\_(content=\aspas{},mimetype=None,status=200, content\_type=None), \_\_delitem\_\_(header), \_\_getitem\_\_(header)\_\_setitem\_\_(header, val), delete\_cookie(key,path=\aspas{/},domain=None), flush(), has\_header(header), set\_cookie, set\_signed\_cookie(key,value,max\_age=None, expires=None,path=\aspas{/},domain=None, secure=None,httponly=True), tell(), write(content)}
	\end{minipage}
  };
  \node[fancytitle, right=10pt] at (box.north west) {Request/Response};
\end{tikzpicture}

%------------ Configurações Personalizadas ---------------------
\begin{tikzpicture}
  \node [mybox] (box){%
	\begin{minipage}{0.3\textwidth} \vspace{0.5em}
	  No Arquivo \textbf{settings.py} colocar em Português: \\
	  \codigo{TIME\_ZONE = 'America/Sao\_Paulo' \\
	  LANGUAGE\_CODE = 'pt-BR'} \\[2mm]
	  No Arquivo \textbf{settings.py} adicionar a pasta static para arquivos CSS, IMAGE e JS: \\
      \codigo{STATIC\_URL = '/static/' \\
      STATIC\_ROOT = os.path.join(BASE\_DIR, 'static')}
	\end{minipage}
  };
  \node[fancytitle, right=10pt] at (box.north west) {Configurações Personalizadas};
\end{tikzpicture}

%------------ Configurações Personalizadas ---------------------
\begin{tikzpicture}
  \node [mybox] (box){%
	\begin{minipage}{0.3\textwidth} \vspace{0.5em}
	  \textbf{Oficial BR}: \\
	  \url{https://docs.djangoproject.com/pt-br/3.1/intro/}
	  \textbf{Django Girls}: \\ 
	  \url{https://tutorial.djangogirls.org/pt/} \\
	  \textbf{Real Python}: \\ 
	  \url{https://realpython.com/get-started-with-django-1/} \\
	  \textbf{Tutorials Point}: \\
	  \url{https://www.tutorialspoint.com/django}
	\end{minipage}
  };
  \node[fancytitle, right=10pt] at (box.north west) {Bons Tutoriais};
\end{tikzpicture}

Página do Django: \url{https://www.djangoproject.com/}

\end{multicols*}
\end{document}